\section{Рабочий проект}
\subsection{Классы, используемые при разработке сайта}

Можно выделить следующий список классов и их методов, использованных при разработке web-приложения (таблица \ref{class:table}).

\renewcommand{\arraystretch}{0.8} % уменьшение расстояний до сетки таблицы
\begin{xltabular}{\textwidth}{|X|p{2.5cm}|>{\setlength{\baselineskip}{0.7\baselineskip}}p{4.85cm}|>{\setlength{\baselineskip}{0.7\baselineskip}}p{4.85cm}|}
\caption{Описание классов View, используемых в приложении\label{class:table}}\\
\hline \centrow \setlength{\baselineskip}{0.7\baselineskip} Название класса & \centrow \setlength{\baselineskip}{0.7\baselineskip} Модуль, к которому относится класс & \centrow Описание класса & \centrow Методы \\
\hline \centrow 1 & \centrow 2 & \centrow 3 & \centrow 4\\ \hline
\endfirsthead
\caption*{Продолжение таблицы \ref{class:table}}\\
\hline \centrow 1 & \centrow 2 & \centrow 3 & \centrow 4\\ \hline
\finishhead
View & Главный модуль & View – основной класс для взаимодействия сервера с веб-страницами. Абстрактный класс, от которого наследуются все остальные view-классы. & template = '' – переменная для шаблонов дочерних классов.
def get(self, environ) – метод обработки GET-запросов.
def post(self, environ) – метод обработки POST-запросов.
def handle(self, environ) – метод для определения типа запроса (GET или POST) и вызова соответствующего метода.  \\
\hline CreateTopicView & Главный модуль & CreateTopicView – класс для обработки страницы создания темы. & template = 'templates/create_topic.html' – путь к HTML шаблону.
def post(self, environ) – метод для обработки формы создания темы и сохранения данных в базе данных. Возвращает JSON ответ о статусе операции.  \\

\hline ErrorView & Главный модуль & ErrorView – класс для обработки страницы ошибок. & template = template = 'templates/404.html' – путь к HTML шаблону для страницы ошибки.  \\


\hline GetMessagesView & Главный модуль & GetMessagesView – класс для получения сообщений с сервера. & def get(self, environ) – метод для обработки GET-запроса, получения сообщений из базы данных и возврата их в формате JSON.  \\

\hline IndexView & Главный модуль & IndexView – класс для отображения главной страницы. & template = 'templates/index.html' – путь к HTML шаблону главной страницы.  \\

\hline GetTopicView & Главный модуль & GetTopicView – класс для получения списка тем с сервера. & def get(self, environ) – метод для обработки GET-запроса, получения списка тем из базы данных и возврата их в формате JSON.  \\

\hline GetUsersView & Главный модуль & GetUsersView – класс для получения списка пользователей с сервера. & def get(self, environ) – метод для обработки GET-запроса, получения списка пользователей из базы данных и возврата их в формате JSON.  \\


\hline RegisterView & Главный модуль & RegisterView – класс для обработки страницы регистрации. & template = 'templates/register.html' – путь к HTML шаблону страницы регистрации. \\

\hline SendMessageView & Главный модуль & SendMessageView – класс для обработки отправки сообщений. & def post(self, environ) – метод для обработки формы отправки сообщений и сохранения данных в базе данных. Возвращает JSON ответ о статусе операции.   \\

\hline StaticView & Главный модуль & StaticView – класс для обработки статических файлов. & def get(self, environ) – метод для получения статического файла по запросу и возврата его содержимого с соответствующим MIME-типом.   \\

\hline TemplateView & Главный модуль & TemplateView – абстрактный класс для обработки страниц с HTML шаблонами. & template = '' – переменная для пути к HTML шаблону. def get(self, environ) – метод для рендеринга HTML шаблона и возврата его содержимого.   \\


\hline TopicPageView & Главный модуль & TopicPageView – класс для отображения страницы конкретной темы. & template = 'templates/topic_page.html' – путь к HTML шаблону страницы темы. \\



\hline TopicsView & Главный модуль & TopicsView – класс для отображения страницы со списком тем. & template = 'templates/topics.html' – путь к HTML шаблону страницы со списком тем. \\

\hline UserView & Главный модуль & UserView – класс для отображения страницы профиля пользователя. & template = 'templates/user.html' – путь к HTML шаблону страницы профиля.  \\

\end{xltabular}
\renewcommand{\arraystretch}{1.0} % восстановление сетки

\newpage
\subsection{Системное тестирование разработанного web-сайта}

На рисунке \ref{main:image} представлена главная страница сайта форума «8chan».

\begin{figure}[H] % H - рисунок обязательно здесь, или переносится, оставляя пустоту
\center{\includegraphics[width=1\linewidth]{main1}}
\caption{Главная страница сайта «8chan»}
\label{main:image}
\end{figure}

На рисунке \ref{menu:image} представлен динамический вывод тем.

\begin{figure}[ht]
\center{\includegraphics[width=1\linewidth]{menu}}
\caption{Динамический вывод тем}
\label{menu:image}
\end{figure}

На рисунке \ref{enter:image} представлен ввод данных для публикации темы.

\begin{figure}[ht]
\center{\includegraphics[width=1\linewidth]{enter}}
\caption{Ввод данных для публикации темы}
\label{enter:image}
\end{figure}
