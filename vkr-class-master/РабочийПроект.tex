\section{Рабочий проект}
\subsection{Классы, используемые при разработке сайта}

Можно выделить следующий список классов и их методов, использованных при разработке web-приложения (таблица \ref{class:table}).

\renewcommand{\arraystretch}{0.8} % уменьшение расстояний до сетки таблицы
\begin{xltabular}{\textwidth}{|X|p{2.5cm}|>{\setlength{\baselineskip}{0.7\baselineskip}}p{4.85cm}|>{\setlength{\baselineskip}{0.7\baselineskip}}p{4.85cm}|}
\caption{Описание классов View, используемых в приложении\label{class:table}}\\
\hline \centrow \setlength{\baselineskip}{0.7\baselineskip} Название класса & \centrow \setlength{\baselineskip}{0.7\baselineskip} Модуль, к которому относится класс & \centrow Описание класса & \centrow Методы \\
\hline \centrow 1 & \centrow 2 & \centrow 3 & \centrow 4\\ \hline
\endfirsthead
\caption*{Продолжение таблицы \ref{class:table}}\\
\hline \centrow 1 & \centrow 2 & \centrow 3 & \centrow 4\\ \hline
\finishhead
View & Главный модуль & View – главный класс для взаимодействия сервера со страницами web-приложения. Является родительским (абстрактным) классом для всех View классов, таких как, например, IndexView (отображение html шаблона), StaticView (обработка статических файлов), GetTopicView (GET - запрос) и т.п. & template = '' - 
переменная для дочерних классов View связанных с отображением html шаблонов
def get(self, environ) - 
функция для дочерних классов View связанных с GET запросами \\
\hline CreateTopicView & Главный модуль & CreateTopicView – Класс для работы со страницей создания темы & template = 'templates/create\_topic.html' - переменная содержащая в себе адрес html файла.
def get(self, environ) - функция для отображения страницы создания темы (возвращает функцию render\_template(template\_name=self.template, context={})) \\
\hline IndexView & Главный модуль & IndexView – Класс для работы с главной страницей форума & template = 'templates/index.html' - переменная содержащая в себе адрес html файла.
def get(self, environ) - функция для отображения главной страницы (возвращает функцию render\_template(template\_name=self.template, context={})) \\
\hline RegisterView & Главный модуль & RegisterView – Класс для работы со страницей для регистрации & template = 'templates/register.html' - переменная содержащая в себе адрес html файла.
def get(self, environ) - функция для отображения страницы регистрации (возвращает функцию render\_template(template\_name=self.template, context={})) \\
\hline TopicPageView & Главный модуль & TopicPageView – Класс для работы со страницей определенной темы & template = 'templates/topic\_page.html' - переменная содержащая в себе адрес html файла.
def get(self, environ) - функция для отображения страницы определенной темы (возвращает функцию render\_template(template\_name=self.template, context={})) \\
\hline TopicsView & Главный модуль & TopicsView – Класс для работы со страницей тем & template = 'templates/topics.html' - переменная содержащая в себе адрес html файла.
def get(self, environ) - функция для отображения страницы тем (возвращает функцию render\_template(template\_name=self.template, context={})) \\
\hline UserView & Главный модуль & UserView – Класс для работы со страницей личного профиля & template = 'templates/user.html' - переменная содержащая в себе адрес html файла.
def get(self, environ) - функция для отображения страницы профиля (возвращает функцию render\_template(template\_name=self.template, context={})) \\
\hline RegisterView & Главный модуль & GetTopicView – Класс для работы со страницей для регистрации & template = 
def get(self, environ) - функция для передачи в БД GET запроса (возвращает функцию get\_topics\_from\_db() \\
\hline RegisterView & Главный модуль & GetUsersView – Класс для работы со страницей для регистрации & template = 
def get(self, environ) - функция для передачи в БД GET запроса (возвращает функцию get\_users\_from\_db() \\

\end{xltabular}
\renewcommand{\arraystretch}{1.0} % восстановление сетки

\newpage
\subsection{Системное тестирование разработанного web-сайта}

На рисунке \ref{main:image} представлена главная страница сайта форума «8chan».

\begin{figure}[H] % H - рисунок обязательно здесь, или переносится, оставляя пустоту
\center{\includegraphics[width=1\linewidth]{main1}}
\caption{Главная страница сайта «8chan»}
\label{main:image}
\end{figure}

На рисунке \ref{menu:image} представлен динамический вывод тем.

\begin{figure}[ht]
\center{\includegraphics[width=1\linewidth]{menu}}
\caption{Динамический вывод тем}
\label{menu:image}
\end{figure}

На рисунке \ref{enter:image} представлен ввод данных для публикации темы.

\begin{figure}[ht]
\center{\includegraphics[width=1\linewidth]{enter}}
\caption{Ввод данных для публикации темы}
\label{enter:image}
\end{figure}
