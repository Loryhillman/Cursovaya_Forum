\section{Рабочий проект}
\subsection{Классы, используемые при разработке сайта}

Можно выделить следующий список классов и их методов, использованных при разработке web-приложения (таблицы \ref{class:table0} - \ref{class:table13}).

\renewcommand{\arraystretch}{0.8} % уменьшение расстояний до сетки таблицы
\begin{xltabular}{\textwidth}{|X|p{6cm}|>{\setlength{\baselineskip}{0.7\baselineskip}}p{6cm}|>{\setlength{\baselineskip}{0.7\baselineskip}}>{\raggedright\arraybackslash}X|}
	\caption{Описание класса View\label{class:table0}}\\
	\hline \centrow \setlength{\baselineskip}{1\baselineskip} Название класса & \centrow \setlength{\baselineskip}{0.7\baselineskip} Описание класса & \centrow Методы \\
	\hline \centrow 1 & \centrow 2 & \centrow 3\\ \hline
	\endfirsthead
	\hline \centrow 1 & \centrow 2 & \centrow 3\\ \hline
	\finishhead
	View & View – основной класс для взаимодействия сервера с веб-страницами. Абстрактный класс, от которого наследуются все остальные view-классы. & template = '' – переменная для шаблонов дочерних классов.
	def get(self, environ) – метод обработки GET-запросов.
	def post(self, environ) – метод обработки POST-запросов.
	def handle(self, environ) – метод для определения типа запроса (GET или POST) и вызова соответствующего метода.  \\
\end{xltabular}
\renewcommand{\arraystretch}{1.0} % восстановление сетки



\renewcommand{\arraystretch}{0.8} % уменьшение расстояний до сетки таблицы
\begin{xltabular}{\textwidth}{|X|p{6cm}|>{\setlength{\baselineskip}{0.7\baselineskip}}p{6cm}|>{\setlength{\baselineskip}{0.7\baselineskip}}>{\raggedright\arraybackslash}X|}
	\caption{Описание класса ErrorView\label{class:table1}}\\
	\hline \centrow \setlength{\baselineskip}{1\baselineskip} Название класса & \centrow \setlength{\baselineskip}{0.7\baselineskip} Описание класса & \centrow Методы \\
	\hline \centrow 1 & \centrow 2 & \centrow 3\\ \hline
	\endfirsthead
	\hline \centrow 1 & \centrow 2 & \centrow 3\\ \hline
	\finishhead
	ErrorView & ErrorView – класс для обработки страницы ошибок. & template = template = 'templates/404.html' – путь к HTML шаблону для страницы ошибки.
	  \\
\end{xltabular}
\renewcommand{\arraystretch}{1.0} % восстановление сетки

\renewcommand{\arraystretch}{0.8} % уменьшение расстояний до сетки таблицы
\begin{xltabular}{\textwidth}{|X|p{6cm}|>{\setlength{\baselineskip}{0.7\baselineskip}}p{6cm}|>{\setlength{\baselineskip}{0.7\baselineskip}}>{\raggedright\arraybackslash}X|}
	\caption{Описание класса GetMessagesView\label{class:table2}}\\
	\hline \centrow \setlength{\baselineskip}{1\baselineskip} Название класса & \centrow \setlength{\baselineskip}{0.7\baselineskip} Описание класса & \centrow Методы \\
	\hline \centrow 1 & \centrow 2 & \centrow 3\\ \hline
	\endfirsthead
	\hline \centrow 1 & \centrow 2 & \centrow 3\\ \hline
	\finishhead
	GetMessagesView & GetMessagesView – класс для получения сообщений с сервера. & def get(self, environ) – метод для обработки GET-запроса, получения сообщений из базы данных и возврата их в формате JSON.
	\\
\end{xltabular}
\renewcommand{\arraystretch}{1.0} % восстановление сетки


\renewcommand{\arraystretch}{0.8} % уменьшение расстояний до сетки таблицы
\begin{xltabular}{\textwidth}{|X|p{6cm}|>{\setlength{\baselineskip}{0.7\baselineskip}}p{6cm}|>{\setlength{\baselineskip}{0.7\baselineskip}}>{\raggedright\arraybackslash}X|}
	\caption{Описание класса IndexView\label{class:table3}}\\
	\hline \centrow \setlength{\baselineskip}{1\baselineskip} Название класса & \centrow \setlength{\baselineskip}{0.7\baselineskip} Описание класса & \centrow Методы \\
	\hline \centrow 1 & \centrow 2 & \centrow 3\\ \hline
	\endfirsthead
	\hline \centrow 1 & \centrow 2 & \centrow 3\\ \hline
	\finishhead
	IndexView & IndexView – класс для отображения главной страницы. & template = 'templates/index.html' – путь к HTML шаблону главной страницы.
	\\
\end{xltabular}
\renewcommand{\arraystretch}{1.0} % восстановление сетки

\renewcommand{\arraystretch}{0.8} % уменьшение расстояний до сетки таблицы
\begin{xltabular}{\textwidth}{|X|p{6cm}|>{\setlength{\baselineskip}{0.7\baselineskip}}p{6cm}|>{\setlength{\baselineskip}{0.7\baselineskip}}>{\raggedright\arraybackslash}X|}
	\caption{Описание класса GetTopicView\label{class:table4}}\\
	\hline \centrow \setlength{\baselineskip}{1\baselineskip} Название класса & \centrow \setlength{\baselineskip}{0.7\baselineskip} Описание класса & \centrow Методы \\
	\hline \centrow 1 & \centrow 2 & \centrow 3\\ \hline
	\endfirsthead
	\hline \centrow 1 & \centrow 2 & \centrow 3\\ \hline
	\finishhead
	GetTopicView & GetTopicView – класс для получения списка тем с сервера. & def get(self, environ) – метод для обработки GET-запроса, получения списка тем из базы данных и возврата их в формате JSON.
	\\
\end{xltabular}
\renewcommand{\arraystretch}{1.0} % восстановление сетки

\renewcommand{\arraystretch}{0.8} % уменьшение расстояний до сетки таблицы
\begin{xltabular}{\textwidth}{|X|p{6cm}|>{\setlength{\baselineskip}{0.7\baselineskip}}p{6cm}|>{\setlength{\baselineskip}{0.7\baselineskip}}>{\raggedright\arraybackslash}X|}
	\caption{Описание класса GetUsersView\label{class:table5}}\\
	\hline \centrow \setlength{\baselineskip}{1\baselineskip} Название класса & \centrow \setlength{\baselineskip}{0.7\baselineskip} Описание класса & \centrow Методы \\
	\hline \centrow 1 & \centrow 2 & \centrow 3\\ \hline
	\endfirsthead
	\hline \centrow 1 & \centrow 2 & \centrow 3\\ \hline
	\finishhead
	GetUsersView & GetUsersView – класс для получения списка пользователей с сервера. & def get(self, environ) – метод для обработки GET-запроса, получения списка пользователей из базы данных и возврата их в формате JSON.
	\\
\end{xltabular}
\renewcommand{\arraystretch}{1.0} % восстановление сетки

\renewcommand{\arraystretch}{0.8} % уменьшение расстояний до сетки таблицы
\begin{xltabular}{\textwidth}{|X|p{6cm}|>{\setlength{\baselineskip}{0.7\baselineskip}}p{6cm}|>{\setlength{\baselineskip}{0.7\baselineskip}}>{\raggedright\arraybackslash}X|}
	\caption{Описание класса RegisterView\label{class:table6}}\\
	\hline \centrow \setlength{\baselineskip}{1\baselineskip} Название класса & \centrow \setlength{\baselineskip}{0.7\baselineskip} Описание класса & \centrow Методы \\
	\hline \centrow 1 & \centrow 2 & \centrow 3\\ \hline
	\endfirsthead
	\hline \centrow 1 & \centrow 2 & \centrow 3\\ \hline
	\finishhead
	RegisterView & RegisterView – класс для обработки страницы регистрации. & template = 'templates/register.html' – путь к HTML шаблону страницы регистрации.
	\\
\end{xltabular}
\renewcommand{\arraystretch}{1.0} % восстановление сетки

\renewcommand{\arraystretch}{0.8} % уменьшение расстояний до сетки таблицы
\begin{xltabular}{\textwidth}{|X|p{6cm}|>{\setlength{\baselineskip}{0.7\baselineskip}}p{6cm}|>{\setlength{\baselineskip}{0.7\baselineskip}}>{\raggedright\arraybackslash}X|}
	\caption{Описание класса SendMessageView\label{class:table7}}\\
	\hline \centrow \setlength{\baselineskip}{1\baselineskip} Название класса & \centrow \setlength{\baselineskip}{0.7\baselineskip} Описание класса & \centrow Методы \\
	\hline \centrow 1 & \centrow 2 & \centrow 3\\ \hline
	\endfirsthead
	\hline \centrow 1 & \centrow 2 & \centrow 3\\ \hline
	\finishhead
	MessageView & MessageView – класс для обработки отправки сообщений. & def post(self, environ) – метод для обработки формы отправки сообщений и сохранения данных в базе данных. Возвращает JSON ответ о статусе операции.
	\\
\end{xltabular}
\renewcommand{\arraystretch}{1.0} % восстановление сетки

\renewcommand{\arraystretch}{0.8} % уменьшение расстояний до сетки таблицы
\begin{xltabular}{\textwidth}{|X|p{6cm}|>{\setlength{\baselineskip}{0.7\baselineskip}}p{6cm}|>{\setlength{\baselineskip}{0.7\baselineskip}}>{\raggedright\arraybackslash}X|}
	\caption{Описание класса StaticView\label{class:table8}}\\
	\hline \centrow \setlength{\baselineskip}{1\baselineskip} Название класса & \centrow \setlength{\baselineskip}{0.7\baselineskip} Описание класса & \centrow Методы \\
	\hline \centrow 1 & \centrow 2 & \centrow 3\\ \hline
	\endfirsthead
	\hline \centrow 1 & \centrow 2 & \centrow 3\\ \hline
	\finishhead
	StaticView & StaticView – класс для обработки статических файлов. & def get(self, environ) – метод для получения статического файла по запросу и возврата его содержимого с соответствующим MIME-типом.
	\\
\end{xltabular}
\renewcommand{\arraystretch}{1.0} % восстановление сетки

\renewcommand{\arraystretch}{0.8} % уменьшение расстояний до сетки таблицы
\begin{xltabular}{\textwidth}{|X|p{6cm}|>{\setlength{\baselineskip}{0.7\baselineskip}}p{6cm}|>{\setlength{\baselineskip}{0.7\baselineskip}}>{\raggedright\arraybackslash}X|}
	\caption{Описание класса TemplateView\label{class:table39}}\\
	\hline \centrow \setlength{\baselineskip}{1\baselineskip} Название класса & \centrow \setlength{\baselineskip}{0.7\baselineskip} Описание класса & \centrow Методы \\
	\hline \centrow 1 & \centrow 2 & \centrow 3\\ \hline
	\endfirsthead
	\hline \centrow 1 & \centrow 2 & \centrow 3\\ \hline
	\finishhead
	TemplateView & TemplateView – абстрактный класс для обработки страниц с HTML шаблонами. & template = '' – переменная для пути к HTML шаблону. def get(self, environ) – метод для рендеринга HTML шаблона и возврата его содержимого.
	\\
\end{xltabular}
\renewcommand{\arraystretch}{1.0} % восстановление сетки

\renewcommand{\arraystretch}{0.8} % уменьшение расстояний до сетки таблицы
\begin{xltabular}{\textwidth}{|X|p{6cm}|>{\setlength{\baselineskip}{0.7\baselineskip}}p{6cm}|>{\setlength{\baselineskip}{0.7\baselineskip}}>{\raggedright\arraybackslash}X|}
	\caption{Описание класса TopicPageView\label{class:table10}}\\
	\hline \centrow \setlength{\baselineskip}{1\baselineskip} Название класса & \centrow \setlength{\baselineskip}{0.7\baselineskip} Описание класса & \centrow Методы \\
	\hline \centrow 1 & \centrow 2 & \centrow 3\\ \hline
	\endfirsthead
	\hline \centrow 1 & \centrow 2 & \centrow 3\\ \hline
	\finishhead
	TopicPageView & TopicPageView – класс для отображения страницы конкретной темы. & template = 'templates/topiс\_page' – путь к HTML шаблону страницы темы.
	\\
\end{xltabular}
\renewcommand{\arraystretch}{1.0} % восстановление сетки

\renewcommand{\arraystretch}{0.8} % уменьшение расстояний до сетки таблицы
\begin{xltabular}{\textwidth}{|X|p{6cm}|>{\setlength{\baselineskip}{0.7\baselineskip}}p{6cm}|>{\setlength{\baselineskip}{0.7\baselineskip}}>{\raggedright\arraybackslash}X|}
	\caption{Описание класса TopicsView\label{class:table11}}\\
	\hline \centrow \setlength{\baselineskip}{1\baselineskip} Название класса & \centrow \setlength{\baselineskip}{0.7\baselineskip} Описание класса & \centrow Методы \\
	\hline \centrow 1 & \centrow 2 & \centrow 3\\ \hline
	\endfirsthead
	\hline \centrow 1 & \centrow 2 & \centrow 3\\ \hline
	\finishhead
	TopicsView & TopicsView – класс для отображения страницы со списком тем. & template = 'templates/topics.html' – путь к HTML шаблону страницы со списком тем.
	\\
\end{xltabular}
\renewcommand{\arraystretch}{1.0} % восстановление сетки

\renewcommand{\arraystretch}{0.8} % уменьшение расстояний до сетки таблицы
\begin{xltabular}{\textwidth}{|X|p{6cm}|>{\setlength{\baselineskip}{0.7\baselineskip}}p{6cm}|>{\setlength{\baselineskip}{0.7\baselineskip}}>{\raggedright\arraybackslash}X|}
	\caption{Описание класса UserView\label{class:table12}}\\
	\hline \centrow \setlength{\baselineskip}{1\baselineskip} Название класса & \centrow \setlength{\baselineskip}{0.7\baselineskip} Описание класса & \centrow Методы \\
	\hline \centrow 1 & \centrow 2 & \centrow 3\\ \hline
	\endfirsthead
	\hline \centrow 1 & \centrow 2 & \centrow 3\\ \hline
	\finishhead
	UserView & UserView – класс для отображения страницы профиля пользователя. & template = 'templates/user.html' – путь к HTML шаблону страницы профиля.
	\\
\end{xltabular}
\renewcommand{\arraystretch}{1.0} % восстановление сетки

\renewcommand{\arraystretch}{0.8} % уменьшение расстояний до сетки таблицы
\begin{xltabular}{\textwidth}{|X|p{6cm}|>{\setlength{\baselineskip}{0.7\baselineskip}}p{6cm}|>{\setlength{\baselineskip}{0.7\baselineskip}}>{\raggedright\arraybackslash}X|}
	\caption{Описание класса Response\label{class:table13}}\\
	\hline \centrow \setlength{\baselineskip}{1\baselineskip} Название класса & \centrow \setlength{\baselineskip}{0.7\baselineskip} Описание класса & \centrow Методы \\
	\hline \centrow 1 & \centrow 2 & \centrow 3\\ \hline
	\endfirsthead
	\hline \centrow 1 & \centrow 2 & \centrow 3\\ \hline
	\finishhead
	Response & Response - Этот класс представляет объект ответа сервера, который включает в себя данные, тип контента и HTTP-код ответа. & def get\_data(self) Метод возвращает данные для ответа в виде байтовой строки, закодированных в UTF-8, если self.data является строкой. Если self.data не является строкой (например, это уже байтовые данные), метод возвращает их без изменений.
	\\
\end{xltabular}
\renewcommand{\arraystretch}{1.0} % восстановление сетки




\newpage
\subsection{Системное тестирование разработанного web-сайта}

На рисунке \ref{main:image} представлена главная страница сайта форума «8chan».

\begin{figure}[H] % H - рисунок обязательно здесь, или переносится, оставляя пустоту
\center{\includegraphics[width=1\linewidth]{main1}}
\caption{Главная страница сайта «8chan»}
\label{main:image}
\end{figure}

На рисунке \ref{menu:image} представлен динамический вывод тем.

\begin{figure}[H]
\center{\includegraphics[width=1\linewidth]{menu}}
\caption{Динамический вывод тем}
\label{menu:image}
\end{figure}

\newpage
На рисунке \ref{enter:image} представлен ввод данных для публикации темы.

\begin{figure}[H]
\center{\includegraphics[width=1\linewidth]{enter}}
\caption{Ввод данных для публикации темы}
\label{enter:image}
\end{figure}
