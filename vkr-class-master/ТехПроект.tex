\section{Технический проект}
\subsection{Общая характеристика организации решения задачи}

Необходимо спроектировать и разработать сайт-форум для обсуждения на нём различных тем.


Веб-сайт — это комплекс взаимосвязанных электронных страниц, сгруппированных в разделы и доступных через уникальный интернет-адрес в форме www.имя\_сайта.ru. Каждая страница веб-сайта представляет собой текстовый документ, созданный с применением различных языков программирования, таких как HTML, CSS, JavaScript и другие.

\subsection{Обоснование выбора технологии проектирования}

На сегодняшний день информационный рынок, поставляющий программные решения в выбранной сфере, предлагает множество продуктов, позволяющих достигнуть поставленной цели – разработки web-сайта.

\subsubsection{Описание используемых технологий и языков программирования}

В процессе разработки web-сайта используются программные средства и языки программирования. Каждое программное средство и каждый язык программирования применяется для круга задач, при решении которых они необходимы.

\subsubsection{Язык программирования Python}

Python – это интерпретируемый, высокоуровневый язык программирования с динамической типизацией, который обеспечивает простоту в использовании, читаемость кода и эффективное взаимодействие с другими языками и системами. Основное предназначение данного языка – это создание серверной части веб-приложения, обеспечивая взаимодействие с веб-сервером. Это позволяет обрабатывать HTTP-запросы, управлять сессиями и осуществлять обработку данных от клиентов, а также взаимодействовать с базой данных.

\subsubsection{Язык программирования JavaScript}

\paragraph{Достоинства языка JavaScript}

JavaScript – это высокоуровневый, интерпретируемый язык программирования, который широко используется для создания интерактивных и динамичных веб-сайтов. Разработанный Netscape, JavaScript стал ключевым компонентом веб-технологий, позволяя веб-страницам взаимодействовать с пользователем, изменять содержимое и обеспечивать богатый пользовательский опыт. JavaScript используется для создания интерактивных элементов пользовательского интерфейса на веб-сайтах форумов. Это включает в себя динамическую подгрузку данных, анимации, обработку событий и обновление содержимого страницы без ее перезагрузки. JavaScript может обеспечивать валидацию данных, введенных пользователями в формах, перед отправкой на сервер. Также он применяется для асинхронного взаимодействия с сервером, например, для отправки комментариев, без необходимости перезагрузки всей страницы. JavaScript позволяет динамически обновлять содержимое форума в реальном времени. Это может включать в себя добавление новых тем, обновление сообщений, и другие изменения, которые пользователь может видеть без необходимости обновления всей страницы. JavaScript используется для взаимодействия с серверной частью, осуществляя запросы к API форума на Python. Это включает передачу данных между клиентом и сервером, а также обновление интерфейса в соответствии с полученными данными.

\paragraph{Недостатки языка Javascript}

JavaScript может вести себя по-разному в различных браузерах, что создает проблемы с кросс-браузерной совместимостью. Некоторые функции могут работать по-разному или вовсе отсутствовать в различных браузерах, что требует дополнительного тестирования и учета особенностей при разработке. JavaScript выполняется на стороне клиента, и поэтому пользователи могут иметь доступ к исходному коду и внести изменения. Это может создавать уязвимости безопасности, такие как внедрение вредоносного кода или модификация данных на стороне клиента. По соображениям безопасности JavaScript имеет ограниченный доступ к файловой системе клиентского устройства. Это может затруднить реализацию некоторых функций, таких как загрузка и сохранение файлов на стороне клиента. Средства отладки JavaScript в браузере могут быть менее мощными и удобными, чем средства отладки на сервере. Это может затруднить выявление и устранение ошибок в коде JavaScript. Обработка ошибок в JavaScript иногда может быть неочевидной из-за асинхронной природы языка. Это может затруднить выявление и отладку ошибок, особенно в сложных веб-приложениях.

\subsection{Диаграмма взаимодействия и схема обмена данными между представлениями}

Диаграмма взаимодействия описывает физическое представление системы и устанавливает связи между ее программными представлениями, включая исходный и исполняемый код. Она служит инструментом для определения архитектуры системы, выявляя зависимости между программными элементами. На диаграмме взаимодействия используются графические элементы, такие как представления, интерфейсы и их взаимосвязи.

Каждое представление должно быть активировано в контексте веб-страницы. В процессе вызова представления, данные передаются из веб-страницы в само представление.

Файл index.py представляет собой пример простого веб-приложения на Python, использующего фреймворк WSGI (Web Server Gateway Interface). В этом файле происходит импорт необходимых библиотек, классов и функций, таких как CGI, Waitress для веб-сервера, различные представления (views) и функции для работы с файлами и базой данных.



Словарь urls связывает URL-пути с соответствующими представлениями. Когда поступает запрос, код использует этот словарь для определения, какой обработчик использовать для конкретного URL. Основная функция app обрабатывает POST-запрос для загрузки файла, извлекает данные из формы, вставляет изображение в базу данных и возвращает ответ в формате JSON. Обработка GET-запроса осуществляется путем логики определения и вызова соответствующего представления, а также обработки статических файлов, установки MIME-типа и кодировки для ответа, установки заголовков ответа и возврата данных в ответе.


\subsubsection{Структура серверной части}
Структура серверной части мессенджера представлена следующим образом:

\begin{figure}[H]
	\centering
	\includegraphics[width=0.6\linewidth]{images/Server_diag}
	\caption{Диаграмма компонентов сервера}
	\label{fig:serverdiag}
\end{figure}

index.py - запускает веб-сервер Waitress, который обрабатывает HTTP-запросы с помощью функции маршрутизации route\_request. Он определяет приложение WSGI и запускает сервер для обработки входящих запросов и отправки ответов клиентам.

route\_request.py - обрабатывает входящие HTTP-запросы, сопоставляя их с маршрутами и вызывая соответствующие представления для генерации ответов. Он затем возвращает сгенерированные ответы клиенту, установив соответствующий код состояния и заголовки.

routes.py - определяет список маршрутов для веб-приложения, связывая URL-паттерны с соответствующими классами представлений. Он импортирует необходимые представления и указывает маршрут для обработки ошибок через ErrorView, который используется для всех несоответствующих URL.

templates\_view - определяют базовый класс View и различные классы представлений для обработки HTTP-запросов. Они реализуют методы для обработки GET и POST запросов, обрабатывают данные формы и взаимодействуют с базой данных, возвращая соответствующие ответы клиентам.

render\_template.py - определяет функцию render\_template, которая загружает HTML-шаблон из файла, подставляет в него данные из переданного словаря контекста и возвращает сгенерированный HTML-код в виде строки.

response.py - определяет класс Response, который формирует HTTP-ответы с заданными данными, типом контента и статусным кодом. Класс содержит методы для кодирования данных в UTF-8 и форматирования статусного кода с сообщением, обеспечивая удобное представление и передачу ответов в веб-приложении.

register.html и register.js - проверяет авторизацию пользователя через cookie и скрывает определенные элементы навигации для неавторизованных пользователей. Он также добавляет обработчик формы регистрации, который отправляет данные на сервер и обрабатывает ответ, перенаправляя пользователя на главную страницу при успешной регистрации.

login.html и login.js - для авторизации пользователей на веб-сайте. Он позволяет пользователям вводить свои учетные данные (логин и пароль) в форму, отправлять запрос на сервер для проверки этих данных среди зарегистрированных пользователей, и в случае успешной авторизации сохраняет информацию о пользователе в cookie и localStorage, что позволяет сайту опознавать пользователя и предоставлять доступ к защищенным ресурсам или персонализированным функциям.

app.js - играет ключевую роль в аутентификации пользователей, управлении интерфейсом навигации и обеспечении безопасного доступа к функциональности веб-приложения.


\subsubsection{Структура веб-клиента}
Структура веб-клиента мессенджера представлена следующим образом:

\begin{figure}[H]
	\centering
	\includegraphics[width=0.4\linewidth]{images/klient_diag}
	\caption{Диаграмма компонентов веб-клиента}
	\label{fig:klientdiag}
\end{figure}

\subsubsection{Диаграмма классов}
На рисунке \ref{fig:classdiag} изображена диаграмма классов для моего проекта веб-мессенджера. Данная диаграмма визуализирует взаимодействие между различными классами, представляющими пользователей, сообщения, чаты, элементы управления и другие основные компоненты системы.


\begin{figure}
	\centering
	\includegraphics[width=1.05\linewidth]{images/class_diag}
	\caption{Диаграмма классов}
	\label{fig:classdiag}
\end{figure}

TemplateView - эта группа классов представляет собой реализацию шаблонизации для веб-приложения. TemplateView служит базовым классом для создания видов, которые отображают содержимое HTML-шаблонов с данными из контекста.

View - этот базовый класс предоставляет общий интерфейс для обработки HTTP-запросов. Методы get и post позволяют переопределять поведение для обработки GET и POST запросов соответственно, а метод handle определяет, какой метод вызывать в зависимости от типа запроса (GET или POST), что делает его основой для создания различных видов (views) веб-приложения.

StaticView - этот класс-наследник View предназначен для обслуживания статических файлов веб-приложения. При получении GET запроса он открывает запрошенный файл из файловой системы, определяет MIME-тип и возвращает содержимое файла в виде HTTP-ответа с соответствующим кодом состояния (200 для успешного запроса или 404 в случае отсутствия файла).

CreateTopicView - этот класс-наследник TemplateView предназначен для обработки POST запросов при создании новой темы в форуме. При получении POST запроса извлекает данные формы, передает их в функцию create\_topic для сохранения в базе данных и возвращает успешный HTTP-ответ в формате JSON с кодом состояния 200.

CreateUserView - этот класс-наследник View обрабатывает POST запросы для создания нового пользователя. При получении POST запроса извлекает данные формы, передает их в функцию create\_user для регистрации пользователя в системе и возвращает успешный HTTP-ответ в формате JSON с кодом состояния 200.

GetMessagesView - этот класс-наследник View обрабатывает GET запросы для получения сообщений по указанной теме. При получении GET запроса извлекает параметры запроса, вызывает функцию get\_messages\_with\_username для извлечения данных из базы данных и возвращает полученные данные в формате JSON с кодом состояния 200.

GetTopicView - этот класс-наследник View обрабатывает GET запросы для получения списка тем из базы данных. При получении GET запроса вызывает функцию get\_topics\_from\_db для извлечения данных о темах и возвращает их в формате JSON с кодом состояния 200.

GetUsersView - этот класс-наследник View обрабатывает GET запросы для получения списка пользователей из базы данных. При получении GET запроса вызывает функцию get\_users\_from\_db для извлечения данных о пользователях и возвращает их в формате JSON с кодом состояния 200.

SendMessageView - этот класс-наследник View обрабатывает POST запросы для отправки сообщений в указанную тему. При получении POST запроса извлекает данные формы, передает их в функцию send\_message для сохранения сообщения в базе данных и возвращает успешный HTTP-ответ в формате JSON с кодом состояния 200.




\subsection{Содержание информационных блоков. Основные сущности}











Проанализировав требования, можно выделить три основных сущности:
\begin{itemize}
\item "<Пользователь">;
\item "<Сообщение">;
\item "<Тема">.
\end{itemize}

В состав сущности "<Сообщение"> можно включить атрибуты, представленные в таблице \ref{message:table}.

\begin{xltabular}{\textwidth}{|l|l|p{1.7cm}|X|}
	\caption{Атрибуты сущности "<Сообщение">\label{message:table}}\\ \hline
	\centrow Поле & \centrow Тип & \centrow Обяза\-тельное & \centrow Описание \\ \hline
	\thead{1} & \thead{2} & \centrow 3 & \centrow 4 \\ \hline
	\endfirsthead

	id\_message & Integer & true & Уникальный идентификатор сообщения \\ \hline 
	id\_user\_message & Integer & true & Уникальный идентификатор сообщения пользователя \\ \hline 
	topic\_id & Integer & true & Уникальный идентификатор темы, в которой написано сообщение \\ \hline 
	message\_text & String & true & Текст сообщения \\ \hline 
\end{xltabular}

В состав сущности "<Пользователь"> можно включить атрибуты, представленные в таблице \ref{user:table}.

\begin{xltabular}{\textwidth}{|l|lp{1.7cm}|X|}
	\caption{Атрибуты сущности "<Пользователь">\label{user:table}}\\ \hline
	\centrow Поле & \centrow Тип & \centrow Обяза\-тельное & \centrow Описание \\ \hline
	\thead{1} & \thead{2} & \centrow 3 & \centrow 4 \\ \hline
	\endfirsthead
	id\_user & Integer & true & Уникальный идентификатор пользователя \\ \hline 
	login & String & true & Имя пользователя \\ \hline 
	password & String & true & Пароль пользователя \\ \hline 
\end{xltabular}

В состав сущности "<Тема"> можно включить атрибуты, представленные в таблице \ref{topic:table}.

\begin{xltabular}{\textwidth}{|l|l|p{1.7cm}|X|}
	\caption{Атрибуты сущности "<Тема">\label{topic:table}}\\ \hline
	\centrow Поле & \centrow Тип & \centrow Обяза\-тельное & \centrow Описание \\ \hline
	\thead{1} & \thead{2} & \centrow 3 & \centrow 4 \\ \hline
	\endfirsthead
	id\_message & Integer & true & Уникальный идентификатор сообщения \\ \hline 
	id\_user\_message & Integer & true & Уникальный идентификатор сообщения пользователя \\ \hline 
	topic\_id & Integer & true & Уникальный идентификатор темы, в которой написано сообщение \\ \hline 
	message\_text & String & true & Текст сообщения \\ \hline 
\end{xltabular} 
