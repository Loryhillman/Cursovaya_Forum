\section{Анализ предметной области}
\subsection{Характеристика предприятия и его деятельности. Тест длинного заголовка, не должен содержать переносы}

Технология трёхмерной печати появилась в конце 80-х гг. ХХ в. Пионером в этой области является компания 3D Systems, которая разработала первую коммерческую стереолитографическую машину – SLA – \linebreak Stereolithography Apparatus (1986 г.). До середины 90-х гг. она использовалась в научно-исследовательской и опытноконструкторской деятельности, связанной с оборонной промышленностью. Первые лазерные машины – сначала стереолитографические (SLA-машины), затем порошковые (SLS-машины) – были чрезмерно дороги, а выбор модельных материалов скромный. Широкое распространение цифровых технологий в области проектирования (CAD), моделирования и расчётов (CAE) и механообработки (CAM) стимулировало развитие технологий 3D-печати. 

Термин "<аддитивные технологии"> (АТ) означает изготовление изделия путем добавления. АТ являются новыми методами в производстве различного рода изделий. Применение данных технологий допускает как создание изделий с нуля, так и обработку уже имеющихся. Сегодня трудно найти отрасль производства, где бы ни применялись 3D-принтеры: с их помощью изготавливаются детали самолетов, космических аппаратов, подводных лодок, инструменты, протезы и др.

АТ предполагают изготовление (построение) физического объекта (детали) методом послойного нанесения материала, в отличие от традиционных методов формирования детали, за счёт удаления материала из массива заготовки.

АТ охватывают все новые сферы деятельности человека. Дизайнеры, архитекторы, кондитеры, археологи, астрономы, палеонтологи и представители других профессий используют 3D-принтеры для реализации различных идей и проектов. 

Деятельность отраслевого интегратора "<Русатом -- Аддитивные технологии"> охватывает все составляющие аддитивного рынка: производство 3D-принтеров, выпуск оборудования для создания порошков, разработка программного обеспечения и организация центров аддитивного производства. Сегодня аддитивные технологии внедряются в самые сложные и наукоемкие отрасли: атомную промышленность, аэрокосмическую индустрию, медицину, автомобилестроение и многие другие. Применение АТ решает задачи по снижению стоимости, сокращения срока изготовления изделий и обеспечение высокой персонализации деталей.
\subsection{Аддитивные технологии, их классификация}

Основное преимущество АТ состоит в том, что прототип создается за один прием, а исходными данными для него служит геометрическая модель детали. В итоге отпадает необходимость в планировании последовательности технологических процессов, специальном оборудовании для обработки материалов, транспортировке от станка к станку и т. д.

Экструзионная печать. Включает такие методы, как послойное наплавление и многоструйная печать.

Стереолитография. Стереолитографические принтеры используют специальные жидкие материалы, называемые "<фотополимерными смолами">.

Ламинирование. Слои материала наклеиваются друг на друга и обрезаются по контурам цифровой модели с помощью лазера или лезвия. 
