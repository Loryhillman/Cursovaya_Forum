\section{Анализ предметной области}

Создание веб-сайта форума на языке программирования Python без использования готовых веб-фреймворков представляет собой сложную задачу, требующую глубокого анализа не только технических аспектов веб-разработки, но и особенностей функционирования форумов в онлайн-среде. Форумы остаются важной частью онлайн-культуры, предоставляя место для обмена идеями, решения проблем, обсуждения интересов и формирования виртуальных сообществ. Анализ их роли в онлайн-сообществах позволяет определить ключевые функциональные требования к создаваемому веб-сайту форума. Python, как язык программирования, обладает удобством синтаксиса и богатым набором библиотек. Анализ его применения для создания веб-приложений без фреймворков позволяет определить преимущества и сложности этого подхода. Анализ существующих веб-фреймворков подчеркивает их важность в упрощении процесса создания веб-приложений. Однако, отказ от их использования в контексте создания форума требует внимательного рассмотрения потенциальных выгод и вызовов такого решения. Безопасность важна для веб-приложений, особенно при обработке личных данных пользователей на форуме. Анализ предметной области включает в себя рассмотрение методов обеспечения безопасности данных и защиты от распространенных атак. Анализ требований к адаптивному дизайну и удобству использования веб-сайта форума учитывает разнообразные устройства, на которых пользователи могут просматривать и взаимодействовать с контентом. Это включает в себя создание удобного интерфейса без использования готовых клиентских фреймворков. Создание веб-сайта форума без фреймворков также предоставляет возможность глубокого анализа и оптимизации производительности, включая работу с базой данных, обработку запросов и оптимизацию передачи данных между клиентом и сервером.
