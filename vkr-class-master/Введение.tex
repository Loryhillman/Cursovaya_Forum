\section*{ВВЕДЕНИЕ}
\addcontentsline{toc}{section}{ВВЕДЕНИЕ}

В современной эпохе цифровых технологий веб-сайты форумов остаются одним из ключевых средств общественного обмена и взаимодействия в виртуальном пространстве. Создание функционального и эффективного форума представляет собой задачу, требующую глубокого понимания аспектов веб-разработки и использования инструментов программирования для достижения поставленных целей.

Цель данной курсовой работы заключается в разработке веб-сайта форума на языке программирования Python без применения готовых веб-фреймворков (за исключением фреймворков для серверной части). Это предоставляет уникальную возможность глубже понять принципы работы веб-приложений, архитектурные особенности и механизмы безопасности, лежащие в основе функционирования современных интерактивных онлайн-платформ.

В рамках данного исследования будут рассмотрены различные аспекты создания веб-форума: от проектирования базы данных и реализации серверной части до разработки динамичного пользовательского интерфейса. Основной акцент будет сделан на использовании Python для написания серверной логики, при этом исключая привлечение сторонних веб-фреймворков для обеспечения основной функциональности.

В ходе работы будут решены задачи, связанные с безопасностью данных, оптимизацией производительности, а также созданием удобного и интуитивно понятного интерфейса для пользователей. Это исследование не только предоставит взгляд на процесс разработки веб-сайта форума, но также выявит преимущества и ограничения подхода без использования веб-фреймворков на стороне клиента.

Создание функционального форума на Python без привлечения готовых фреймворков представляет собой интеллектуальное исследование, которое позволит лучше понять особенности веб-разработки и вносить свой вклад в область создания веб-приложений.

\emph{Цель настоящей работы} – работы является создание и исследование веб-сайта форума на языке программирования Python, обращая особое внимание на реализацию серверной части без использования готовых веб-фреймворков, за исключением фреймворков, предназначенных для обработки запросов и взаимодействия с базой данных. \emph{следующие задачи:}
\begin{itemize}
\item определение структуры базы данных для хранения информации о темах, сообщениях, пользователях и других сущностях;
\item написание серверной логики на языке Python для обработки запросов, создания, редактирования и удаления тем и сообщений, а также управления пользователями;
\item создание динамичного пользовательского интерфейса с использованием HTML, CSS и JavaScript без использования готовых клиентских фреймворков;
\item исследование и оптимизация алгоритмов работы веб-сайта для обеспечения высокой производительности.
\end{itemize}

\emph{Структура и объем работы.} Отчет состоит из введения, 4 разделов основной части, заключения, списка использованных источников, 2 приложений. Текст выпускной квалификационной работы равен \formbytotal{page}{страниц}{е}{ам}{ам}.

\emph{Во введении} сформулирована цель работы, определены задачи разработки, представлен обзор структуры работы и краткое содержание каждого из разделов.

\emph{В первом разделе} на этапе описания технической характеристики предметной области, включает в себя сбор информации о текущей деятельности компании, для которой разрабатывается веб-сайт.

\emph{Во втором разделе} на этапе технического задания, содержит требования к созданию веб-сайта.

\emph{В третьем разделе} на стадии технического проектирования, представляет собой описание проектных решений для веб-сайта.

\emph{В четвертом разделе} представлен перечень классов и их методов, используемых при разработке веб-сайта, а также проведено тестирование созданного сайта.

В заключении представлены основные результаты работы, полученные в процессе разработки веб-сайта.


В приложении А представлен графический материал.
В приложении Б представлены фрагменты исходного кода. 
